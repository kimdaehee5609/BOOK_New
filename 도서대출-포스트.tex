%	------------------------------------------------------------------------------
%
%		도서 대출
%
%		작성 : 
%				2022년 
%				8월 
%				05일 
%				금요일
%				첫 작업
%
%

%	\documentclass[25pt, a1paper]{tikzposter}
%	\documentclass[25pt, a0paper, landscape]{tikzposter}
%	\documentclass[25pt, a1paper ]{tikzposter}
	\documentclass[	20pt, 
							a1paper, 
							portrait, %
							margin=0mm, %
							innermargin=10mm,  		%
%							blockverticalspace=4mm, %
							colspace=5mm, 
							subcolspace=0mm
							]{tikzposter}


%	\documentclass[25pt, a1paper]{tikzposter}
%	\documentclass[25pt, a1paper]{tikzposter}
%	\documentclass[25pt, a1paper]{tikzposter}

% 	12pt  14pt 17pt  20pt  25pt
%
%	a0 a1 a2
%
%	landscape  portrait
%

	%% Tikzposter is highly customizable: please see
	%% https://bitbucket.org/surmann/tikzposter/downloads/styleguide.pdf

	%	========================================================== 	Package
		\usepackage{kotex}						% 한글 사용


%% Available themes: see also
%% https://bitbucket.org/surmann/tikzposter/downloads/themes.pdf
%	\usetheme{Default}
%	\usetheme{Rays}
%	\usetheme{Basic}
	\usetheme{Simple}
%	\usetheme{Envelope}
%	\usetheme{Wave}
%	\usetheme{Board}
%	\usetheme{Autumn}
%	\usetheme{Desert}

%% Further changes to the title etc is possible
%	\usetitlestyle{Default}			%
%	\usetitlestyle{Basic}				%
%	\usetitlestyle{Empty}				%
%	\usetitlestyle{Filled}				%
%	\usetitlestyle{Envelope}			%
%	\usetitlestyle{Wave}				%
%	\usetitlestyle{verticalShading}	%


%	\usebackgroundstyle{Default}
%	\usebackgroundstyle{Rays}
%	\usebackgroundstyle{VerticalGradation}
%	\usebackgroundstyle{BottomVerticalGradation}
%	\usebackgroundstyle{Empty}

%	\useblockstyle{Default}
%	\useblockstyle{Basic}
%	\useblockstyle{Minimal}		% 이것은 간단함
%	\useblockstyle{Envelope}		% 
%	\useblockstyle{Corner}		% 사각형
%	\useblockstyle{Slide}			%	띠모양  
	\useblockstyle{TornOut}		% 손그림모양


	\usenotestyle{Default}
%	\usenotestyle{Corner}
%	\usenotestyle{VerticalShading}
%	\usenotestyle{Sticky}

%	\usepackage{fontspec}
%	\setmainfont{FreeSerif}
%	\setsansfont{FreeSans}

%	------------------------------------------------------------------------------ 제목

	\title{도서 대출}

	\author{ 2022년 32주차 8월8일 ~ 8월 14일  }

%	\institute{서영엔지니어링}
%	\titlegraphic{\includegraphics[width=7cm]{IMG_1934}}

	%% Optional title graphic
	%\titlegraphic{\includegraphics[width=7cm]{IMG_1934}}
	%% Uncomment to switch off tikzposter footer
	% \tikzposterlatexaffectionproofoff

\begin{document}

	\maketitle

	\begin{columns}

		\column{0.5}

%	------------------------------------------------------------------------------ 옹기종기 도서관
			\block{■  옹기종기 도서관  }
			{
					\setlength{\leftmargini}{4em}
					\setlength{\labelsep} {1em}
				\begin{LARGE}
					\begin{itemize}
					\item 	2022/08/12 금 ~ 2022/08/26 금
					\item 	[32 01] 그림으로 배우는 역학 기초
					\item 	[32 02] 색다른 물리학(상)
					\item 	[32 03] 인체 해부학으로 본 동물도감
					\item 	[32 04] 그리스 로마 신화
					\item 	[32 05] 의사가 읽어주는 그리스 로마신화
					\end{itemize}
				\end{LARGE}
			}


%%	------------------------------------------------------------------------------ 중앙 도서관
%			\block{■  중앙 도서관  }
%			{
%					\setlength{\leftmargini}{4em}
%					\setlength{\labelsep} {1em}
%				\begin{LARGE}
%					\begin{itemize}
%					\item 	2022/08/05 금 ~ 2022/08/19 금
%					\item 	[01] 
%					\item 	[02] 
%					\item 	[03]  
%					\item 	[04]  
%					\item 	[05]  
%					\end{itemize}
%				\end{LARGE}
%			}



%%	------------------------------------------------------------------------------ 남구 도서관
%			\block{■  남구 도서관 }
%			{
%					\setlength{\leftmargini}{4em}
%					\setlength{\labelsep} {1em}
%				\begin{LARGE}
%					\begin{itemize}
%					\item 	2022/08/05 금 ~ 2022/08/19 금
%					\item 	[31 06]  무료 디자인 제작하기
%					\item 	[31 07]  산색
%					\item 	[31 08]  다시  확률 통계
%					\item 	[31 09]  (조셉 필라테스의)필라테브 바이블
%					\item 	[31 10] 필라테스 바이블  
%					\end{itemize}
%				\end{LARGE}
%			}

%	------------------------------------------------------------------------------ 내리새라 도서관
			\block{■  내리새라 도서관 }
			{
					\setlength{\leftmargini}{4em}
					\setlength{\labelsep} {1em}
				\begin{LARGE}
					\begin{itemize}
					\item 	2022/08/12 금 ~ 2022/08/26 금
					\item 	[32 06] 틱닛한 지구별 모든 생명에게
					\item 	[32 07] 기계는 어떻게 생각하고 학습하는가
					\item 	[32 08] 따라 쓰기의 기적
					\item 	[32 09] 걷기의 세계
					\item 	[32 10] 과식하지 않는 삶
					\end{itemize}
				\end{LARGE}
			}



%%	------------------------------------------------------------------------------ 수정 도서관
%			\block{■  수정 도서관 }
%			{
%					\setlength{\leftmargini}{4em}
%					\setlength{\labelsep} {1em}
%				\begin{LARGE}
%					\begin{itemize}
%					\item 	2022/08/07 일 ~ 2022/08/21 일
%					\item 	[31 11] 통기타 명곡집
%					\item 	[31 12] 코드송 통기타 200
%					\item 	[31 13]  허리 좀 펴고 삽시다
%					\end{itemize}
%				\end{LARGE}
%			}


%	------------------------------------------------------------------------------ 영도 도서관
			\block{■  영도 도서관 }
			{
					\setlength{\leftmargini}{4em}
					\setlength{\labelsep} {1em}
				\begin{LARGE}
					\begin{itemize}
					\item 	2022/08/14 일 ~ 2022/08/28 일
					\item 	[32 11]	나도 칼리바를 잘 치면 소원이 없겠네
					\item 	[32 12]	사마타와 위빠사나
					\item 	[32 13]	과학으로 풀어보는 음악의 비밀
					\item 	[32 14]	이이화의 이야기 한국불교사
					\item 	[32 15]	크리에이터 박터틀의 작곡 독학,가이드북
					\end{itemize}
				\end{LARGE}
			}


%%	------------------------------------------------------------------------------ 구덕 도서관
%			\block{■  구덕 도서관 }
%			{
%					\setlength{\leftmargini}{4em}
%					\setlength{\labelsep} {1em}
%				\begin{LARGE}
%					\begin{itemize}
%					\item 	2022/08/07 일 ~ 2022/08/21 일
%					\item 	[31 16]  프로 일잘러의 슬기로운 노션 활용법
%					\item 	[31 17]  한개의 기쁨이 천개의 슬픔을 이긴다
%					\item 	[31 18]  붓다의 영적 돌봄
%					\item 	[31 19]  한역으로 읽는 알아차림의 확립 수행
%					\item 	[31 20]  된다 하루5분 노션 활용법
%					\end{itemize}
%				\end{LARGE}
%			}
%
%






	%	====== ====== ====== ====== ====== 
		\column{0.5}


%%	------------------------------------------------------------------------------ 금정 도서관
%			\block{■  금정 도서관}
%			{
%					\setlength{\leftmargini}{4em}
%					\setlength{\labelsep} {1em}
%				\begin{LARGE}
%					\begin{itemize}
%					\item 	2022/08/05 금 ~ 2022/08/19 금
%					\item 	[01] 
%					\item 	[02] 
%					\item 	[03]  
%					\item 	[04]  
%					\item 	[05]  
%					\end{itemize}
%				\end{LARGE}
%			}
%




	\end{columns}




\end{document}


		\begin{huge}
		\end{huge}

		\begin{LARGE}
		\end{LARGE}

		\begin{Large}
		\end{Large}

		\begin{large}
		\end{large}

