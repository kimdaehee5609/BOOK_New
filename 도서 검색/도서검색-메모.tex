%	-------------------------------------------------------------------------------
%
%		도서 검색
%
%		작성 : 
%				2022년 
%				8월 
%				05일 
%				금요일
%				첫 작업

%
%
%
%
%
%	-------------------------------------------------------------------------------

%\documentclass[10pt,xcolor=pdftex,dvipsnames,table]{beamer}
%\documentclass[10pt,blue,xcolor=pdftex,dvipsnames,table,handout]{beamer}
%\documentclass[14pt,blue,xcolor=pdftex,dvipsnames,table,handout]{beamer}
\documentclass[aspectratio=1610,17pt,xcolor=pdftex,dvipsnames,table,handout]{beamer}

		% Font Size
		%	default font size : 11 pt
		%	8,9,10,11,12,14,17,20
		%
		% 	put frame titles 
		% 		1) 	slideatop
		%		2) 	slide centered
		%
		%	navigation bar
		% 		1)	compress
		%		2)	uncompressed
		%
		%	Color
		%		1) blue
		%		2) red
		%		3) brown
		%		4) black and white	
		%
		%	Output
		%		1)  	[default]	
		%		2)	[handout]		for PDF handouts
		%		3) 	[trans]		for PDF transparency
		%		4)	[notes=hide/show/only]

		%	Text and Math Font
		% 		1)	[sans]
		% 		2)	[sefif]
		%		3) 	[mathsans]
		%		4)	[mathserif]


		%	---------------------------------------------------------	
		%	슬라이드 크기 설정 ( 128mm X 96mm )
		%	---------------------------------------------------------	
%			\setbeamersize{text margin left=2mm}
%			\setbeamersize{text margin right=2mm}

	%	========================================================== 	Package
		\usepackage{kotex}						% 한글 사용
		\usepackage{amssymb,amsfonts,amsmath}	% 수학 수식 사용
		\usepackage{color}					%
		\usepackage{colortbl}					%


	%		========================================================= 	note 옵션인 
	%			\setbeameroption{show only notes}
		

	%		========================================================= 	Theme

		%	---------------------------------------------------------	
		%	전체 테마
		%	---------------------------------------------------------	
		%	테마 명명의 관례 : 도시 이름
%			\usetheme{default}			%
%			\usetheme{Madrid}    		%
%			\usetheme{CambridgeUS}    	% -red, no navigation bar
%			\usetheme{Antibes}			% -blueish, tree-like navigation bar

		%	----------------- table of contents in sidebar
			\usetheme{Berkeley}		% -blueish, table of contents in sidebar
									% 개인적으로 마음에 듬

%			\usetheme{Marburg}			% - sidebar on the right
%			\usetheme{Hannover}		% 왼쪽에 마크
%			\usetheme{Berlin}			% - navigation bar in the headline
%			\usetheme{Szeged}			% - navigation bar in the headline, horizontal lines
%			\usetheme{Malmoe}			% - section/subsection in the headline

%			\usetheme{Singapore}
%			\usetheme{Amsterdam}

		%	---------------------------------------------------------	
		%	색 테마
		%	---------------------------------------------------------	
%			\usecolortheme{albatross}	% 바탕 파란
%			\usecolortheme{crane}		% 바탕 흰색
%			\usecolortheme{beetle}		% 바탕 회색
%			\usecolortheme{dove}		% 전체적으로 흰색
%			\usecolortheme{fly}		% 전체적으로 회색
%			\usecolortheme{seagull}	% 휜색
%			\usecolortheme{wolverine}	& 제목이 노란색
%			\usecolortheme{beaver}

		%	---------------------------------------------------------	
		%	Inner Color Theme 			내부 색 테마 ( 블록의 색 )
		%	---------------------------------------------------------	

%			\usecolortheme{rose}		% 흰색
%			\usecolortheme{lily}		% 색 안 칠한다
%			\usecolortheme{orchid} 	% 진하게

		%	---------------------------------------------------------	
		%	Outter Color Theme 		외부 색 테마 ( 머리말, 고리말, 사이드바 )
		%	---------------------------------------------------------	

%			\usecolortheme{whale}		% 진하다
%			\usecolortheme{dolphin}	% 중간
%			\usecolortheme{seahorse}	% 연하다

		%	---------------------------------------------------------	
		%	Font Theme 				폰트 테마
		%	---------------------------------------------------------	
%			\usfonttheme{default}		
			\usefonttheme{serif}			
%			\usefonttheme{structurebold}			
%			\usefonttheme{structureitalicserif}			
%			\usefonttheme{structuresmallcapsserif}			



		%	---------------------------------------------------------	
		%	Inner Theme 				
		%	---------------------------------------------------------	

%			\useinnertheme{default}
			\useinnertheme{circles}		% 원문자			
%			\useinnertheme{rectangles}		% 사각문자			
%			\useinnertheme{rounded}			% 깨어짐
%			\useinnertheme{inmargin}			




		%	---------------------------------------------------------	
		%	이동 단추 삭제
		%	---------------------------------------------------------	
%			\setbeamertemplate{navigation symbols}{}

		%	---------------------------------------------------------	
		%	문서 정보 표시 꼬리말 적용
		%	---------------------------------------------------------	
%			\useoutertheme{infolines}


			
	%	---------------------------------------------------------- 	배경이미지 지정
%			\pgfdeclareimage[width=\paperwidth,height=\paperheight]{bgimage}{./fig/Chrysanthemum.jpg}
%			\setbeamertemplate{background canvas}{\pgfuseimage{bgimage}}

		%	---------------------------------------------------------	
		% 	본문 글꼴색 지정
		%	---------------------------------------------------------	
%			\setbeamercolor{normal text}{fg=purple}
%			\setbeamercolor{normal text}{fg=red!80}	% 숫자는 투명도 표시


		%	---------------------------------------------------------	
		%	itemize 모양 설정
		%	---------------------------------------------------------	
%			\setbeamertemplate{items}[ball]
%			\setbeamertemplate{items}[circle]
%			\setbeamertemplate{items}[rectangle]






		\setbeamercovered{dynamic}





		% --------------------------------- 	문서 기본 사항 설정
		\setcounter{secnumdepth}{5} 		% 문단 번호 깊이
		\setcounter{tocdepth}{5} 			% 문단 번호 깊이




% ------------------------------------------------------------------------------
% Begin document (Content goes below)
% ------------------------------------------------------------------------------
	\begin{document}
	

			\title{도서 검색}

			\author{김대희}

			\date{ 	작성 : 2022년 08월 05일 금요일 \\
					수정 : 2022년 08월 15일 월요일  }






	%	==========================================================
	%		개정 이력
	%	----------------------------------------------------------
	%	2022.08.05 	첫 작성
	%	2022.08.15  수정
	%	----------------------------------------------------------


		\setcounter{tocdepth}{1}


	%	==========================================================
	%
	%	----------------------------------------------------------
		\begin{frame}[plain]
		\titlepage
		\end{frame}



%		\begin{frame} [plain]{목차}
		\begin{frame} {목차}
		\tableofcontents
		\end{frame}

	%	========================================================== 	개요
	%		Frame
	%	----------------------------------------------------------
		\part{도서 검색 }
		\frame{\partpage}


		\begin{frame} [plain]{목차}
		\tableofcontents
		\end{frame}
		

	%	 ----------------------------------------------------------
	%	 Frame
	%	 ----------------------------------------------------------
		\section{도서검색}
%		\frame [plain] {\sectionpage}
		

		\begin{frame} [t,plain]
			\begin{block} {도서검색}
			\begin{itemize}
				\item 파일 위치 : 도서검색 ; 도서관
				\item BOOK New.git
				\item 도서 개요
				\item 저자
				\item 목차
				\item 도서관
			\end{itemize}
			\end{block}
		\end{frame}



	%	========================================================== 	2022년 8월
	%		Frame
	%	----------------------------------------------------------
		\part{2022년 8월}
		\frame{\partpage}


		\begin{frame} [plain]{목차}
		\tableofcontents
		\end{frame}
		


	%	 ----------------------------------------------------------
	%	 Frame
	%	 ---------------------------------------------------------- 2022.08.15.월  이이화 }
		\section{2022.08.15.월  이이화 }
%		\frame [plain] {\sectionpage}
		

		\begin{frame} [t,plain]
			\begin{block} {이이화}
우리나라 대표적인 역사학자이자 고전연구가 및 한문학자이다. 1937년에 한학자이자 『주역』의 대가인 야산也山 이달李達의 넷째 아들로 태어났다. 1945년부터 아버지를 따라 대둔산에 들어가 한문 공부를 했으며, 열여섯 살 되던 해부터 부산·여수·광주 등지에서 고학하면서 학교를 다녔다. 그후 서울에서 문학에 관심을 갖고 대학을 다녔으나 중퇴하고 한국학 및 한국사 탐구에 열중했다.
			\end{block}
		\end{frame}

	%	 ----------------------------------------------------------
	%	 Frame
	%	 ----------------------------------------------------------
		\subsection{2022.08.15.월  이이화 }
%		\frame [plain] {\sectionpage}
		
		\begin{frame} [t,plain]
			\begin{block} {이이화}
지은 책으로 『허균의 생각』 『위대한 봄을 만났다』 『이이화의 한 권으로 읽는 한국사』 『한국의 파벌』 『조선후기 정치사상과 사회변동』 『한국사 이야기』(전22권) 『역사 속의 한국불교』 『인물로 읽는 한국사』(전10권) 『전봉준, 혁명의 기록』 등이 있다.

			\end{block}
		\end{frame}

	%	 ----------------------------------------------------------
	%	 Frame
	%	 ----------------------------------------------------------
		\subsection{2022.08.15.월  이이화 : 허균의 생각}
%		\frame [plain] {\sectionpage}
		
		\begin{frame} [t,plain]
			\begin{block} {2022.08.15.월  이이화 : 허균의 생각}
지은 책으로 『허균의 생각』 『위대한 봄을 만났다』 『이이화의 한 권으로 읽는 한국사』 『한국의 파벌』 『조선후기 정치사상과 사회변동』 『한국사 이야기』(전22권) 『역사 속의 한국불교』 『인물로 읽는 한국사』(전10권) 『전봉준, 혁명의 기록』 등이 있다.

			\end{block}
		\end{frame}

	%	 ----------------------------------------------------------
	%	 Frame
	%	 ----------------------------------------------------------
		\subsection{2022.08.15.월  이이화 : 위대한 봄을 만났다}
%		\frame [plain] {\sectionpage}
		
		\begin{frame} [t,plain]
			\begin{block} {2022.08.15.월  이이화 : 위대한 봄을 만났다}
			저자 : 이이화\\
			출판사 : 고유서가 	\\
			발행년도 : 2018\\
정관도서관 911-267\\
영도 911-이68위 \\
구덕 911-427 \\
중앙 911-515 \\
수정 : 911-291 \\
기장 : 911-482
			\end{block}
		\end{frame}


	%	 ----------------------------------------------------------
	%	 Frame
	%	 ----------------------------------------------------------
		\subsection{2022.08.15.월  이이화 : 이이화 의 한권으로 읽는 한국사 }
%		\frame [plain] {\sectionpage}
		
		\begin{frame} [t,plain]
			\begin{block} {2022.08.15.월  이이화 : 이이화 의 한권으로 읽는 한국사 }
구덕	:	911-403	\\
남구	:	911-이68이	\\
금정	:	911-이63한e	\\
해운대	:	911-335	\\
			\end{block}
		\end{frame}

	%	 ----------------------------------------------------------
	%	 Frame
	%	 ----------------------------------------------------------
		\subsection{2022.08.15.월  이이화 : 한국의 파벌}
%		\frame [plain] {\sectionpage}
		
		\begin{frame} [t,plain]
			\begin{block} {2022.08.15.월  이이화 : 한국의 파벌}
수정	:	332.6-13 서고	\\
남구	:	332.6-이68한	\\

			\end{block}
		\end{frame}

	%	 ----------------------------------------------------------
	%	 Frame
	%	 ----------------------------------------------------------
		\subsection{2022.08.15.월  이이화 : 조선후기 정치사상과 사회변동}
%		\frame [plain] {\sectionpage}
		
		\begin{frame} [t,plain]
			\begin{block} {2022.08.15.월  이이화 : 조선후기 정치사상과 사회변동}
시민	:	911.05-62 서고	\\
중앙	:	911.057-ㅇ848ㅈ 서고	\\

			\end{block}
		\end{frame}

	%	 ----------------------------------------------------------
	%	 Frame
	%	 ----------------------------------------------------------
		\subsection{2022.08.15.월  이이화 : 한국사 이야기 전22권}
%		\frame [plain] {\sectionpage}
		
		\begin{frame} [t,plain]
			\begin{block} {2022.08.15.월  이이화 : 한국사 이야기 전22권 }
수정		:	911-321	\\
금정		:	911-9	\\
해운대	:	911-296	\\
기장		:	911-5	\\
정관		:	911-183	\\
영도		:	911.008-이68한	\\

			\end{block}
		\end{frame}

	%	 ----------------------------------------------------------
	%	 Frame
	%	 ----------------------------------------------------------
		\subsection{2022.08.15.월  이이화 : 역사속의 한국불교 }
%		\frame [plain] {\sectionpage}
		
		\begin{frame} [t,plain]
			\begin{block} {2022.08.15.월  이이화 : 역사속의 한국불교 }
시민	:	220.911-48	\\
중앙	:	220.911-12서고	\\
기장	:	220.911-6	\\

			\end{block}
		\end{frame}


	%	 ----------------------------------------------------------
	%	 Frame
	%	 ---------------------------------------------------------- 인물로 읽는 한국사
		\subsection{2022.08.15.월  이이화 : 인물로 읽는 한국사 전10권}
%		\frame [plain] {\sectionpage}
		
		\begin{frame} [t,plain]
			\begin{block} {2022.08.15.월  이이화 : 인물로 읽는 한국사 전10권}

			\end{block}
		\end{frame}

	%	 ----------------------------------------------------------
	%	 Frame
	%	 ---------------------------------------------------------- 전봉준의 혁명의 기록 }
		\subsection{2022.08.15.월  이이화 : 전봉준의 혁명의 기록 }
%		\frame [plain] {\sectionpage}
		
		\begin{frame} [t,plain]
			\begin{block} {2022.08.15.월  이이화 : 인물로 전봉준의 혁명의 기록 }
중앙	:	911.059-173	\\
수정	:	911.059-100	\\
구덕	:	911.059-116	\\
남구	:	911.059-이68전	\\
해운대	:	991.1-149	\\
기장	:	991.1-214	\\
영도	:	911.059-이68전	\\
내리새라	:	큰글자 911.059-1	\\
			\end{block}
		\end{frame}


	%	 ----------------------------------------------------------
	%	 Frame
	%	 ----------------------------------------------------------
		\section{9월  }
%		\frame [plain] {\sectionpage}
		

		\begin{frame} [t,plain]
			\begin{block} {9월 }
			\begin{itemize}
					\item [05] 토 	백중기도 화향 지장재일 발열체크
					\item [17] 목 	범어사 초하루법회 발열체크
					\item 자비의 일일찻집 행사 
			\end{itemize}
			\end{block}
		\end{frame}



% ------------------------------------------------------------------------------ ------------------------------------------------------------------------------ ------------------------------------------------------------------------------
% End document
% ------------------------------------------------------------------------------ ------------------------------------------------------------------------------ ------------------------------------------------------------------------------
\end{document}


	%	----------------------------------------------------------
	%		Frame
	%	----------------------------------------------------------
		\begin{frame} [c]
%		\begin{frame} [b]
%		\begin{frame} [t]
		\frametitle{감리 보고서}
		\end{frame}						

